\documentclass{article}
\usepackage[utf8]{inputenc}
\usepackage{amsmath}

\begin{document}

%--------------------------------------------------------------------------------
\section{First example}

The well known Pythagorean theorem \(x^2 + y^2 = z^2\) was
proved to be invalid for other exponents.
Meaning the next equation has no integer solutions:

\[ x^n + y^n = z^n \]


%--------------------------------------------------------------------------------

\section{Second example}

In physics, the mass-energy equivalence is stated by the equation $E=mc^2$, discovered in 1905 by Albert Einstein.

The mass-energy equivalence is described by the famous equation
$$E=mc^2$$
discovered in 1905 by Albert Einstein.
In natural units ($c$ = 1), the formula expresses the identity
\begin{equation}
E=m
\end{equation}

\section{Third example}

This is a simple math expression \(\sqrt{x^2+1}\) inside text.
And this is also the same:
\begin{math}
\sqrt{x^2+1}
\end{math}
but by using another command.

This is a simple math expression without numbering
\[\sqrt{x^2+1}\]
separated from text.

This is also the same:
\begin{displaymath}
\sqrt{x^2+1}
\end{displaymath}

%-----teoria dei segnali--------------------

formula per la rappresentazione dell'energia di un segnale:
$$\int_{-\infty}^{\infty} |f(x)|^2$$

formula per la rappresentazione della potenza media di un segnale:
$$\int_{-\infty}^{\infty} |f(x)|^2$$

serie di fourier di un segnale:

$$\int_{-T}^{T} |f(x) e^{-j2\pi f}$$


$$\lim_{T\to\infty} \frac{1}{2T} \int_{-T}^{T} |x|^2 dx$$

$$\lim_{x\to\infty} f(x)$$

\ldots and this:
\begin{equation*}
\sqrt{x^2+1}
\end{equation*}

\end{document}
